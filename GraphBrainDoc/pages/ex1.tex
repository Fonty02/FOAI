\section{Exercise 1}

\subsection{Overview}

This document provides a detailed description of the updates performed. The modifications have been structured into two main sections: firstly, the upload of various entities to specific classes, and secondly, the improvements proposed for the the interface.

\subsection{Data Upload Details}


\begin{itemize}
    \item \textbf{Metal Slug Series - Main Games}: Approximately 10 main titles from the Metal Slug series  \textit{RETROCOMPUTING $\rightarrow$ VIDEOGAME}
    \item \textbf{Flight Simulator Series}: Around 10 flight simulation games.  \textit{RETROCOMPUTING $\rightarrow$ VIDEOGAME}
    \item \textbf{Street Fighter Series}: Roughly 8 distinct titles.  \textit{RETROCOMPUTING $\rightarrow$ VIDEOGAME}
    \item \textbf{Dragon Ball Games}: About 15 games.  \textit{RETROCOMPUTING $\rightarrow$ VIDEOGAME}
    \item \textbf{Pro Evolution Soccer Series}: Nearly 35 games including both current titles and their historical predecessors.  \textit{RETROCOMPUTING $\rightarrow$ VIDEOGAME}
    \item \textbf{Console Games}: Approximately 5 devices.  \textit{RETROCOMPUTING $\rightarrow$ CONSOLE}
    \item \textbf{Technology Vendors}: Details for 5 companies.  \textit{RETROCOMPUTING $\rightarrow$ COMPANY}
    \item \textbf{Peripheral Devices}: Information for about 15 mouse and keyboard devices.  \textit{RETROCOMPUTING $\rightarrow$ Input Device   (Mouse, Keyboard)}
    \item \textbf{EXPO Events}: A list of approximately 35 events. \textit{RETROCOMPUTING $\rightarrow$ Event}
    \item \textbf{Software Relationships}: For each videogame, a \textit{producedBy} relationship has been established linking the software to the company that developed it.
    \item \textbf{Console Relationships}: For each console, the producing company has been recorded along with associated relationships to already existing consoles.
    \item \textbf{Peripheral Relationships}: For each mouse and keyboard device, the producer has been identified.
    \item \textbf{Geographical Data}: Inclusion of Matera and surrounding cities (approximately 30 locations).
    \item \textbf{Internet Protocols}: Updates include renaming 8 existing protocols and adding around 70 new entries.  \textit{RETROCOMPUTING $\rightarrow$ InternetProtocol}
    \item \textbf{Crapiata}: A traditional dish from Matera, described as a soup made with legumes and vegetables, albeit missing some ingredients.   \textit{FOOD}
    \item \textbf{Culinary Relationships}: Established relevant relationships associated with the aforementioned dish.
\end{itemize}



\subsection{Interface Improvements}

Several adjustments have been made to enhance the user interface:

\begin{itemize}
    \item Incorporation of an HTML Date Type field for the insertion of dates.
    \item Modification of the relationship creation process to allow starting from either the Subject or the Object.
\end{itemize}


\section{Exercise 2}

Here I will provide a brief overview of the changes made the ontology. The modifications are divided by domains and, for each domain, they are divided into two sections: the first one is about the entities and the second one is about the reletionships
\subsection{RETROCOMPUTING}

\subsubsection{Entities}
\begin{itemize}
    \item \textbf{StorageMedium}: I suggest to add a new value for \textit{StorageMedium} called \textit{SolidState}. This value will be used to represent all the solid state storage devices such as SSD, USB pen drive and so on.
    \item \textbf{FPGA}: I suggest to add a new sub-class of \textit{Device} called \textit{FPGA}. This class will be used to represent all the FPGA devices, such as Microchip IGLOO Series
    \item \textbf{Videogame}: Since a videogame can be classified into multiple categories, I suggest to add an attribute to videogame called \textit{Category} that will be a list of categorie such as FPS, Sport, RPG, MOBa and so on. The previously existing sub-classes of \textit{Videogame} have been removed.
    \item \textbf{Preservation Project}: I suggest to add a new class called \textit{PreservationProject} sub-class of \textit{Artifact}. This class will be used to represent all the preservation projects that are related to retrocomputing for example \textit{Internet Archive} or \textit{MAME}. The new attributes are goal (mandatory) and description 
    \item \textbf{Fix}: I suggest to introduce 2 new attributes to \textit{Fix} which are \textit{repairDifficulty} that can assume only 3 values (Beginner, Intermediate, Expert) and \textit{documentationLink} that is a link to the documentation of the fix.
\end{itemize}

\subsubsection{Relationships}
\begin{itemize}
\item \textbf{supports}: I suggest to add this new relationship between \textit{Device} (subject) and \textit{Software} (object). This relationship will be used to represent the software that is supported by a specific device. The attribute is compatibilityNotes
\item \textbf{compatibleWith}: I suggest to add Software (subject) and Component (object). This relationship will be used to represent the software that is compatible with a specific component.
\item \textbf{supports}: I suggest to add this new relationship between \textit{Device} / \textit{OperatingSystem} (subject)and \textit{Software} (object). This relationship will be used to represent the software that is supported by a specific device or operating system. The attribute is compatibilityNotes
\end{itemize}

\subsection{FOOD}
\subsubsection{Entities}
\begin{itemize}
    \item \textbf{Beverage}: I suggest to add a new attribute called \textit{Type} to indicate the type of beverage (alcoholic, non-alcoholic, etc.).
    \item \textbf{Menu Item}: I suggest to add a new attribute called \textit{dietaryInfo} to indicate the dietary information of the menu item (vegan, vegetarian, gluten-free, etc.).
    \item \textbf{SensorialFeature}: Sensorial feature has been removed \footnote{Sensorial Feature may be described as attributes in a relationships without a specific class.}
    \item \textbf{Restaurant}: I suggest to add the attribute \textit{type} to indicate the type of restaurant (fast food, fine dining, etc.).
    \item \textbf{DietaryRestriction}: I suggest to add this new entity to represent the dietary restrictions that can be associated with a food item or menu item. The new attributes are name (mandatory) that can assume fixed values (vegan, vegetarian, gluten-free, etc.) 
    \item \textbf{KitchenTool}: I suggest to add this new entity to represent the kitchen tools that can be used in the preparation of food. The new attributes are name (mandatory)
\end{itemize}

\subsubsection{Relationships}
\begin{itemize}
    \item \textbf{contains}: I suggest to add this new relationship between \textit{FoodBeverage} (subject) and \textit{Nutrient} (object). This relationship will be used to represent the nutrients that are contained in a specific food or beverage. The attribute is quantity (mandatory) that can assume fixed values (low, medium, high).
    \item \textbf{requires}: The subject has been modified from \textit{Artifact} to \textit{KitchenTool}
    \item \textbf{describes}: New attributes have been added to express SensorialFeature 
\end{itemize}

\subsection{OpensScience}

I've added the instruction <import schema "retrocomputing"> to the ontology to import the retrocomputing schema

\subsubsection{Entities}
\begin{itemize}
    \item \textbf{Dataset}: I suggest to add new attributes: creationDate,license,format
    \item \textbf{Environment}: I suggest to add new attributes: type (whose values are Lab, Field or Virtual) and description
    \item \textbf{Author}: I suggest to add \textit{Author} as a sub-class of \textit{Person}
\end{itemize}

\subsection{General}
\subsubsection{Entities}
\begin{itemize}
    \item \textbf{Material}: I suggest to add a new Category called \textit{Material} to represent the materials that can be used to describe Item.
    \item \textbf{Document}: I suggest to add a new attribute called \textit{ToC} to represent the table of contents of the document.
    \item \textbf{Item}: I suggest to add a new attribute called \textit{conditionNotes} to represent the condition of the item
\end{itemize}
\subsubsection{Relationships}
\begin{itemize}
    \item \textbf{madeOf}: I suggest to add this new relationship between \textit{Item} (subject) and \textit{Material} (object)
\end{itemize}

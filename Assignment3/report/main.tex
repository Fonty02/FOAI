\documentclass[12pt,a4paper]{article}

% Packages
\usepackage[utf8]{inputenc}
\usepackage[T1]{fontenc}
\usepackage{amsmath,amssymb}
\usepackage{graphicx}
\usepackage[hidelinks]{hyperref}
\usepackage{authblk} % Package for multiple authors
\usepackage{booktabs} % For better tables
\usepackage{listings} % For code snippets
\usepackage{xcolor} % For colored text

% Title and author information
\title{Language Transformation Analysis: Java to Python Conversion of GraphBrain Domain Package}
\author[1]{Cirilli Davide}
\author[2]{Fontana Emanuele}
\affil[1,2]{Department of Computer Science, Università degli Studi di Bari}


\begin{document}

\maketitle


\tableofcontents
\newpage

\section{Introduction}
This assignment focuses on the implementation of a parser for the Java package \textit{domain} of GraphBrain, recreating its functionality in Python. The domain package constitutes the core model of GraphBrain, containing classes that represent Entities and Relationships, as well as functionality to extract these from \textit{.gbs} files. This language transformation exercise required careful consideration of both language-specific features and object-oriented design principles to ensure functional equivalence between the original Java implementation and the resulting Python code.

The primary objectives of this work were to:
\begin{itemize}
    \item Accurately translate the class hierarchy and inheritance relationships
    \item Preserve method functionality and interface contracts
    \item Adapt Java-specific constructs to idiomatic Python patterns
    \item Maintain the original semantic model of the domain
\end{itemize}


\section{Class Transformations}

\subsection{Attachment Class}
The transformation of the \texttt{Attachment} class from Java to Python required mapping private fields to public attributes while maintaining the intended encapsulation through method interfaces. The Java class's private fields \texttt{progr}, \texttt{extension}, \texttt{description}, and \texttt{fileName} were represented as public attributes in Python, following Python's convention of relying on naming conventions rather than access modifiers.

The constructor logic was preserved in the Python \texttt{\_\_init\_\_} method, initializing all attributes with their corresponding parameters. Additionally, getter methods were implemented with identical names and functionality, with particular attention to the \texttt{getFilename()} method, which replicates the string concatenation logic from the original Java implementation.

\subsection{Tag Abstract Class}Java's abstract class concept was mapped to Python using the \texttt{abc.ABC} base class to enforce abstractness. The protected fields \texttt{name}, \texttt{description}, and \texttt{notes} from the Java implementation were represented as public attributes in Python, initialized to \texttt{None} in the constructor.

All getter and setter methods were faithfully recreated in Python, maintaining the same method signatures and functionality. This approach preserves the original API contract while adapting to Python's conventions regarding attribute visibility.

\subsection{DomainTag Abstract Class}
Building upon the \texttt{Tag} class, the abstract \texttt{DomainTag} class extends the base functionality with domain-specific features. In the Python implementation, \texttt{DomainTag} inherits from both \texttt{Tag} and \texttt{abc.ABC}, preserving the inheritance hierarchy and abstract nature of the class.

The protected field \texttt{domain} from Java was mapped to the attribute \texttt{\_domain} in Python, employing the underscore prefix as a conventional indication of protected status. Corresponding getter and setter methods were implemented to maintain the original interface while adapting to Python's attribute access patterns.

\subsection{Attribute Class}
The \texttt{Attribute} class represents one of the more complex transformations due to its numerous fields, multiple constructors, and diverse methods. Inheriting from \texttt{Tag} in both languages, the Python implementation preserves all public fields (\texttt{mandatory}, \texttt{distinguishing}, \texttt{display}) as public attributes with equivalent names.

The Java class's private collection field \texttt{values} (of type \texttt{List<String>}) was implemented in Python as a typed list (\texttt{List[str]}), leveraging Python's type hints for improved code clarity. Similarly, the \texttt{dataType} field was mapped to \texttt{data\_type} with an \texttt{Optional[str]} type hint, acknowledging that this field might be uninitialized in some contexts.

Java's overloaded constructors were consolidated into a single \texttt{\_\_init\_\_} method in Python, utilizing optional parameters and type checking to replicate the functionality of the multiple Java constructors. This approach maintains the flexibility of the original design while adapting to Python's constructor paradigm.

The comprehensive set of methods in the Java class, including getters, setters, and utility functions, were faithfully recreated in Python with equivalent functionality. Special attention was given to the \texttt{clone()} method, which replicates the deep copying behavior of the Java original.

\subsection{Author Class}
The private fields in Java were mapped to public attributes in Python.

A notable aspect of this transformation was the handling of Java's \texttt{Timestamp} type for the \texttt{creationDate} field, which was mapped to Python's \texttt{datetime} type, providing equivalent functionality while using Python's standard library.

The Java class's implicit default constructor was represented by an \texttt{\_\_init\_\_} method in Python that initializes all attributes to \texttt{None}, preserving the original initialization behavior. Getter and setter methods were implemented with identical names, maintaining the original API contract.

\subsection{Axiom Class}
The \texttt{Axiom} class inherits from \texttt{DomainTag}, preserving the original class hierarchy. Attributes of the Java class were mapped to Python attributes with the same type and name.

The constructor logic was preserved in the \texttt{\_\_init\_\_} method, which calls the superclass constructor before initializing the class's specific attributes.

Particular attention was given to the implementation of Java's \texttt{equals()} and \texttt{hashCode()} methods, which were mapped to Python's special methods \texttt{\_\_eq\_\_()} and \texttt{\_\_hash\_\_()}. These methods maintain the original equality and hashing behavior based on the \texttt{name} attribute, ensuring that collections and comparison operations behave consistently across both languages.

\subsection{UType Class}
The \texttt{UType} class presents an interesting case of inheritance without additional fields or methods. In both Java and Python implementations, \texttt{UType} inherits directly from \texttt{Attribute} without extending the functionality, effectively serving as a specialized type marker.


\subsection{HallUser and HallComparator Classes}
The transformation of \texttt{HallUser} and \texttt{HallComparator} illustrates the adaptation of Java's comparator pattern to Python's comparison protocol. The \texttt{HallUser} class was implemented in Python with equivalent fields and methods, preserving the original data structure and functionality.

The Java \texttt{HallComparator} class, which implements the \texttt{Comparator<HallUser>} interface to define comparison logic, was transformed by integrating its functionality directly into the Python \texttt{HallUser} class through the special methods \texttt{\_\_lt\_\_()} and \texttt{\_\_eq\_\_()}. This approach leverages Python's rich comparison protocol, allowing instances to be naturally sorted according to the original comparison rules (descending order by \texttt{usageStatistic}, then by \texttt{trustIndex}).


\subsection{Instance Class}
Private fields \texttt{type}, \texttt{selectedInstanceId}, \texttt{attributeValues}

\noindent(of type \texttt{Map<String,String>}), and \texttt{shortDescription} were implemented as public attributes in Python, with \texttt{attributeValues} specifically mapped to a typed dictionary (\texttt{Dict[str, str]}).

The constructor logic, which involves building a short description based on attribute values, was faithfully recreated in the Python \texttt{\_\_init\_\_} method. The private Java method \texttt{buildShortDescription} was implemented as a public method in Python, reflecting Python's more relaxed approach to method visibility while preserving the original functionality.

The implementation of Java's \texttt{equals()} method as Python's \texttt{\_\_eq\_\_()} ensures that instance equality is determined by the \texttt{selectedInstanceId} attribute, as in the original Java code, maintaining consistent behavior in collections and comparison operations.

\subsection{Reference Class}
 The private field \texttt{attributes} of type \texttt{Vector<Attribute>} in Java was mapped to \texttt{Optional[List[Attribute]]} in Python, acknowledging both the change in collection type and the possibility of null values.

Java's overloaded constructors were consolidated into a single \texttt{\_\_init\_\_} method with an optional parameter for \texttt{attributes}, defaulting to \texttt{None}. This approach preserves the flexibility of the original design while adapting to Python's constructor paradigm.

The getter and setter methods were implemented with identical names, maintaining the original API contract, and the \texttt{toString()} method was mapped to \texttt{\_\_str\_\_()}, providing a consistent string representation across both languages.

\subsection{Entity Class}

The \texttt{Entity} class, which extends \texttt{DomainTag}, represents one of the central elements in GraphBrain's domain model. Its transformation from Java to Python required particular attention due to the class's complexity, characterized by numerous fields, methods, and hierarchical relationships.

In Python, the class maintains the same inheritance relationship, extending \texttt{DomainTag}. Java's private fields (\texttt{values}, \texttt{graphBrainID}, \texttt{attributes}, \texttt{children}, \texttt{parent}, \texttt{\_abstract}) were mapped to attributes with underscore prefixes in Python, following the convention for indicating protected or private attributes.

A significant aspect of the transformation was adapting the type system: Python type annotations were utilized to improve code readability and maintainability. For example:

\begin{verbatim}
from typing import List, Optional, TYPE_CHECKING

self._values: List[str] = []
self._parent: Optional[Entity] = None
\end{verbatim}

Managing hierarchical relationships between entities required special attention. Methods such as \texttt{getAllAttributes()}, \texttt{getClassPath()}, and \texttt{getSubclassesTree()} were implemented maintaining the same recursive logic as the Java original, but adapted to Python conventions.

Another challenge was implementing the entity comparison system: Java's \texttt{equals()} method was mapped to Python's \texttt{\_\_eq\_\_()} special method, while \texttt{toString()} was mapped to \texttt{\_\_str\_\_()}. This approach ensures that comparison and string representation operations work consistently across both languages.

The hierarchy manipulation methods, including \texttt{addChild()}, \texttt{detach()}, and \texttt{removeAllAttributes()}, were implemented with the same behavior as the original, preserving the consistency of entity relationships during structural modification operations.

The transformation of the \texttt{Entity} class demonstrates the importance of deep understanding of both source and target languages, as differences in programming paradigms require informed decisions to maintain the original semantics while adopting the conventions of the new language.

\subsection{Relationship Class}

The \texttt{Relationship} class, which extends \texttt{Entity}, presented several interesting challenges during the transformation from Java to Python. This class represents connections between entities, forming a central part of GraphBrain's semantic network model.

In the Python implementation, the inheritance relationship with \texttt{Entity} was preserved, while the specialized fields for relationship management were adapted appropriately. The private Java fields \texttt{inverse}, \texttt{references}, \texttt{children}, \texttt{parent}, and \texttt{symmetric} were mapped to attributes with underscore prefixes in Python, following the convention for protected status.

A significant aspect of the transformation was managing the multiple constructors in Java. These were consolidated into a single \texttt{\_\_init\_\_} method in Python using optional parameters:

\begin{verbatim}
def __init__(self, name: str, domain: Optional[str] = None, inverse: Optional[str] = None,
             parent: Optional[Relationship] = None, symmetric: bool = False,
             attributes: Optional[List[Attribute]] = None):
    super().__init__(name, domain)
    self._inverse: Optional[str] = inverse
    self._references: List[Reference] = []
    self._parent: Optional[Relationship] = None
    self._symmetric: bool = symmetric
\end{verbatim}

The transformation required careful type hinting to address potential circular import issues, employing the \texttt{TYPE\_CHECKING} flag from the typing module to avoid runtime import errors while maintaining code clarity.

Managing the hierarchical relationship structure was particularly challenging. Methods like \texttt{setParentRelationship()}, \texttt{addChildrenRelationship()}, and \texttt{isTopRelationship()} were implemented with special attention to maintaining proper bidirectional parent-child references, which required overriding some \texttt{Entity} behaviors.

References management methods, including \texttt{addReference()}, \texttt{getReference()}, and relationship operations like \texttt{getSubj\_Objs()} and \texttt{getObj\_Subjs()}, were implemented with the same functionality as the original. The Python implementation employed set comprehensions and list operations to provide equivalent collection handling to the Java original.

The transformation of the \texttt{Relationship} class demonstrates the complexities involved when mapping classes with both inheritance and composition relationships, requiring careful consideration of type safety and object relationship integrity across the language boundary.

% Inserisci dopo la sezione sulla classe Relationship e prima della Conclusion

\subsection{RelationshipSite Class}

In the Python implementation, the Java private fields \texttt{name}, \texttt{inverse}, \texttt{symmetric}, \texttt{attributes}, and \texttt{relationships} were mapped to public attributes, following Python's convention of relying on naming conventions rather than access modifiers. The constructor logic was preserved in the \texttt{\_\_init\_\_} method, initializing all attributes with their corresponding parameters.


The relationship retrieval methods (\texttt{getSubjects()}, \texttt{getObjects()}, \texttt{getSubj\_Objs()}, and \texttt{getObj\_Subjs()}) were implemented with equivalent functionality as their Java counterparts. The Python implementation maintained the same sorting logic for collections and list operations to provide consistent behavior.


\subsection{RelationTriple Class}
The \texttt{RelationTriple} class represents a fundamental semantic structure in GraphBrain's knowledge representation - a triple consisting of subject, relation, and object instances. This straightforward class was transformed from Java to Python with additions that enhance code quality and maintainability.

In the Python implementation, the Java private fields \texttt{subject}, \texttt{relation}, and \texttt{object} were mapped to public attributes, as is conventional in Python. 


\subsection{Union Class}
The \texttt{Union} class represents a specialized domain tag that contains a set of values. This class extends \texttt{DomainTag} in both Java and Python implementations, preserving the inheritance hierarchy of the original model.

In the Python implementation, the private field \texttt{values} from Java was mapped to a protected attribute \texttt{\_values} in Python, using the underscore prefix to indicate protected status according to Python conventions. Both implementations use a set data structure.

Java's \texttt{equals()} and \texttt{hashCode()} methods were transformed into Python's special methods \texttt{\_\_eq\_\_()} and \texttt{\_\_hash\_\_()}, maintaining the same comparison logic based on the \texttt{name} attribute. This ensures consistent behavior in collections and equality operations:



% Replace subsection 2.14 in your main.txt
\subsection{DomainData Class}
The \texttt{DomainData} class is central to the GraphBrain domain model, orchestrating the loading, parsing, and manipulation of domain information from \textit{.gbs} files. Its responsibilities include managing entities, relationships, attributes, unions, and axioms. Translating this complex Java class into Python required careful consideration of data structures, parsing logic, and object relationships.

The Python implementation maps the extensive field structure of the Java class to appropriately typed attributes, leveraging Python's type hinting system. Key adaptations include:
\begin{itemize}
    \item Java's \texttt{Vector<Attribute>} for \texttt{types} became Python's standard \texttt{List[Attribute]}.
    \item Java's \texttt{HashSet<Union>} for \texttt{unions} corresponds to Python's \texttt{Set[Union]}.
    \item Java's \texttt{Map<String,Vector<String>>} structures (e.g., \texttt{subjRels}) were implemented using Python's \texttt{defaultdict(list)} for conciseness and efficiency.
\end{itemize}

\begin{verbatim}
# Example Python attribute definitions
types: List[Attribute]
unions: Set[Union]
subjRels: Dict[str, List[str]] = defaultdict(list)
\end{verbatim}

Java's multiple overloaded constructors were unified into a single Python \texttt{\_\_init\_\_} method. This method utilizes optional parameters and type checking to provide the same initialization flexibility as the original Java constructors, allowing instantiation from file paths, byte arrays, or with default values.

\begin{verbatim}
    def __init__(self,
    path_or_bytearray: Optional[str | bytes] = None,
    webInfFolder: Optional[str] = None,
    domainName: Optional[str] = None,
    file: Optional[str | Path] = None):
\end{verbatim}

Adapting the XML parsing logic required transitioning from Java's DOM API to Python's standard ElementTree library. The Python implementation includes helper methods (\texttt{parseFile}, \texttt{validateTag}, \texttt{validateTagAttributes}) to replicate the necessary validation and traversal steps performed by the Java code. While ElementTree is idiomatic Python, inherent differences from DOM (e.g., precise handling of comments and whitespace) might exist. Replicating file modification logic (\texttt{substitute}) accurately with ElementTree also presents challenges compared to DOM manipulation.

The recursive parsing methods (\texttt{parseEntities}, \texttt{parseRelationships}, \texttt{parseReferences}) were implemented to construct the domain's hierarchical structure. The correct behavior of these methods relies significantly on the proper implementation of tree manipulation methods (\texttt{addChild}, \texttt{detach}, \texttt{getParent}, \texttt{hasAncestor}, etc.) within the Python \texttt{Entity} and \texttt{Relationship} classes.

Methods that retrieve data based on relationship structures and inheritance (e.g., \texttt{getObjsFromSubjRel}, \texttt{getSubjsFromObjRel}, \texttt{getObjsFromSubRels}) were implemented to query the \textit{direct references} of the relevant relationship(s), consistent with the Java implementation's logic, while also incorporating necessary entity/object inheritance checks using \texttt{findInTree}.

The Python implementation incorporates enhanced data integrity for removal operations. The \texttt{removeEntity} and \texttt{removeRelationship} methods utilize helper functions (\texttt{removeReferencesInvolving}, \texttt{cleanupRelationshipData}) to ensure that auxiliary data structures (like \texttt{subjRels}, \texttt{relObjs}) are comprehensively cleaned up when a primary element is removed, going beyond the simple \texttt{detach} call found in the Java version.

Extensive use of Python's type hints and the addition of comprehensive docstrings enhance the readability and maintainability of the Python \texttt{DomainData} class compared to the original Java code.

% Replace subsection 2.15 in your main.txt
\subsection{RecordData Class}
The \texttt{RecordData} class, much like \texttt{DomainData}, serves as a central repository for domain information but is specifically designed to handle \textit{.gbr} files. Its translation involved similar challenges and adaptations as the \texttt{DomainData} class, ensuring functional equivalence while adopting Pythonic practices.

The Python implementation mirrors the Java \texttt{RecordData}'s field structure, using type annotations for clarity and employing Pythonic collection types. This includes using standard \texttt{List}, \texttt{Set} and \texttt{Dict}, leveraging \texttt{defaultdict(list)} for map-like structures requiring default list initialization.

\begin{verbatim}
# Example Python attribute definitions in RecordData
subjects: List[str]
unions: Set[Union]
subjRel_Objs: Dict[str, List[str]] = defaultdict(list)
\end{verbatim}

Java's overloaded constructors were consolidated into a flexible \texttt{\_\_init\_\_} method, supporting initialization from \textit{.gbr} file paths, byte arrays, or default settings.

The XML parsing logic was adapted from Java's DOM to Python's ElementTree library, with helper methods for file parsing and tag/attribute validation. A specific detail replicated from the Java \texttt{RecordData} is the behavior of the \texttt{\_parseImports} method: even though \texttt{RecordData} primarily handles \textit{.gbr} files, the import mechanism specifically looks for and loads \textit{.gbs} files referenced within the \texttt{<imports>} section, maintaining strict functional equivalence with the observed Java behavior.

The recursive parsing methods (\texttt{\_parseEntities}, \texttt{\_parseRelationships}) were implemented to correctly build the entity and relationship hierarchies, relying on the supporting methods within the Python \texttt{Entity} and \texttt{Relationship} classes.

Attribute inheritance logic for \texttt{properties} and \texttt{propertiesRelation} was implemented correctly. This involves utilizing helper logic (e.g., within \texttt{Entity.getAllAttributes}) that incorporates the necessary stop condition to prevent inheriting attributes from the base "Entity" root node, matching the behavior observed in the Java \texttt{RecordData.propertiesCommon} method.

Data retrieval methods involving relationship traversal and inheritance (e.g., \texttt{getObjsFromSubjRel}, \texttt{getSubjsFromObjRel}) query only the \textit{direct references} associated with the specified relationship(s), aligning with the Java implementation, while still correctly handling entity inheritance checks.

Consistent with the approach in \texttt{DomainData}, the Python \texttt{RecordData} includes comprehensive type hints and docstrings, significantly improving code clarity and documentation. Error handling was also refined to provide more specific and informative feedback compared to the original Java code.

\subsection{TranslatorAPIProlog Class}

The transformation of the \texttt{TranslatorAPIProlog} class represents an interesting case study in adapting specialized domain-specific functionality from Java to Python. This utility class, responsible for converting domain objects into Prolog facts, demonstrates the language-agnostic nature of knowledge representation transformations.

In the Python implementation, the Java class's private fields (\texttt{attributes}, \texttt{attributesRel}, \texttt{domainName}, and \texttt{recur}) were mapped to class attributes with appropriate type annotations:

\begin{verbatim}
serialVersionUID: int = 1  # Placeholder for Java's serialVersionUID
attributes: Optional[Dict[str, List[Attribute]]] = None
attributesRel: Optional[Dict[str, List[Attribute]]] = None
domainName: Optional[str] = None
recur: str = ""
\end{verbatim}

The Java constructor was transformed into a Python \texttt{\_\_init\_\_} method, maintaining the same functionality of generating Prolog facts from a \texttt{DomainData} object. However, the Python implementation adds stronger error checking:

\begin{verbatim}
def __init__(self, domain: DomainData):
    """
    Initializes the TranslatorAPIProlog instance and generates Prolog facts.
    Mirrors the Java constructor.
    """
    if domain.getDomain() is None:
        raise ValueError("Domain name cannot be None")
    # ...
\end{verbatim}

A notable adaptation was the transformation of Java's static methods to Python's \texttt{@staticmethod} decorated functions. This preserved the original design pattern while adapting to Python's syntax:

\begin{verbatim}
@staticmethod
def writeFactsWithId(content: str, id_prefix: str) -> str:
    """
    Adds a unique ID to each fact string.
    Mirrors the Java static method.
    """
    # Implementation
\end{verbatim}

The core string processing methods (\texttt{createEntities}, \texttt{writeEntity}, and \texttt{createRelationships}) underwent significant refactoring to adapt Java's string concatenation approach to Python's formatting capabilities, while maintaining identical output formats:

\begin{verbatim}
# Java: values += "entity(" + entity.getDomain().toLowerCase() + ", " + 
#        entity.getName().toLowerCase() + ").\n";

# Python:
values += f"entity({domain_lower}, {name_lower}).\n"
\end{verbatim}

The Python implementation adds comprehensive error handling and null-safety throughout the string generation process:

\begin{verbatim}
domain_lower = entity.getDomain().lower() if entity.getDomain() else "unknown_domain"
name_lower = entity.getName().lower() if entity.getName() else "unknown_entity"
\end{verbatim}

Methods working with tree structures (\texttt{getClassi}, \texttt{recursiveTree}, and \texttt{recursiveTreeTax}) were preserved but required adaptation to Python's different approach to collections and iteration:

\begin{verbatim}
def recursiveTree(self, node: 'TreeNode') -> List[str]:
    """
    Recursively collects the string representation of all nodes in a subtree.
    Requires a TreeNode implementation.
    """
    classi: List[str] = []
    children = node.getChildren()
    if children:
        for child_node in children:
            classi.append(str(child_node))
            classi.extend(self.recursiveTree(child_node))
    return classi
\end{verbatim}

The private helper method \texttt{createList} in Java was transformed into a Python private method using the underscore prefix convention (\texttt{\_createList}), maintaining Python's naming conventions while preserving the method's utility function status.

A significant improvement in the Python implementation is the addition of comprehensive docstrings, which enhance code readability and provide clear documentation for each method:

\begin{verbatim}
def getDomainName(self) -> Optional[str]:
    """Returns the domain name for this translator."""
    return self.domainName
\end{verbatim}

\section{Additional classes}
\textit{TreeNode} and \textit{DefaultTreeNode} classes don't have a direct equivalent in the Java code, so we created them in Python from scratch.

\subsection{TreeNode Class}
The \texttt{TreeNode} class was created from scratch in Python, inspired by \texttt{org.primefaces.model.TreeNode} from the Java PrimeFaces library. Since GraphBrain's Java implementation utilized this external library class without modifications, we needed to implement an equivalent functionality in our Python conversion.

Our Python implementation maintains all the essential functionality of the original Java class, including methods for managing parent-child relationships, node data, and display state. The class provides a comprehensive API that includes:

\begin{itemize}
    \item Data management: \texttt{getData()}, \texttt{setData()}
    \item Parent-child relationship management: \texttt{getParent()}, \texttt{setParent()}, \texttt{getChildren()}, \texttt{addChild()}, \texttt{removeChild()}
    \item State management: \texttt{isExpanded()}, \texttt{setExpanded()}, \texttt{isSelected()}, \texttt{setSelected()}
    \item Tree navigation: \texttt{getChild\_count()}, \texttt{isLeaf()}
    \item Type information: \texttt{getType()}, \texttt{setType()}
\end{itemize}

Additionally, the Python implementation includes Pythonic enhancements that weren't present in the original Java class:

\begin{itemize}
    \item String representation methods (\texttt{\_\_str\_\_} and \texttt{\_\_repr\_\_}) for easier debugging
    \item Bidirectional relationship management in \texttt{setParent()} to ensure consistency between parent and child nodes
    \item A \texttt{clearParent()} helper method to simplify relationship management
\end{itemize}



\subsection{DefaultTreeNode Class}
The \texttt{DefaultTreeNode} class extends \texttt{TreeNode} and provides a concrete implementation with enhanced constructor flexibility. This class was created to mirror \texttt{org.primefaces.model.DefaultTreeNode} from the Java PrimeFaces library, which was used in the original GraphBrain implementation.


\begin{itemize}
    \item \texttt{DefaultTreeNode()} - Creates an empty node
    \item \texttt{DefaultTreeNode(data)} - Creates a node with specified data
    \item \texttt{DefaultTreeNode(data, parent)} - Creates a node with specified data and parent
    \item \texttt{DefaultTreeNode(type, data, parent)} - Creates a node with specified type, data, and parent
\end{itemize}

The constructor implementation uses Python's optional parameters and type checking to determine which Java constructor variant is being emulated:

\begin{verbatim}
def __init__(self, type_or_data=None, data_or_parent=None, parent=None):
    """
    Initializes a DefaultTreeNode. Mimics PrimeFaces constructors.
    """
    # Logic to determine which constructor variant to emulate
\end{verbatim}

The class also maintains automatic parent-child relationship management, ensuring that when a node is created with a parent, it is automatically added to that parent's children collection - preserving the behavior of the original Java implementation.

\end{document}



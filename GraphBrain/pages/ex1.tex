\subsection{Exercise 1}

\subsubsection{Overview}

This document provides a detailed description of the updates performed. The modifications have been structured into two main sections: firstly, the upload of various entities to specific classes, and secondly, the improvements proposed for the ontology design. Each section is organized into bullet-point lists for clarity.

\subsubsection{Data Upload Details}


\begin{itemize}
    \item \textbf{Metal Slug Series - Main Games}: Approximately 10 main titles from the Metal Slug series  \textit{RETROCOMPUTING $\rightarrow$ VIDEOGAME}
    \item \textbf{Flight Simulator Series}: Around 10 flight simulation games.  \textit{RETROCOMPUTING $\rightarrow$ VIDEOGAME}
    \item \textbf{Street Fighter Series}: Roughly 8 distinct titles.  \textit{RETROCOMPUTING $\rightarrow$ VIDEOGAME}
    \item \textbf{Dragon Ball Games}: About 15 games.  \textit{RETROCOMPUTING $\rightarrow$ VIDEOGAME}
    \item \textbf{Pro Evolution Soccer Series}: Nearly 35 games including both current titles and their historical predecessors.  \textit{RETROCOMPUTING $\rightarrow$ VIDEOGAME}
    \item \textbf{Console Games}: Approximately 5 devices.  \textit{RETROCOMPUTING $\rightarrow$ CONSOLE}
    \item \textbf{Technology Vendors}: Details for 5 companies.  \textit{RETROCOMPUTING $\rightarrow$ COMPANY}
    \item \textbf{Peripheral Devices}: Information for about 15 mouse and keyboard devices.  \textit{RETROCOMPUTING $\rightarrow$ Input Device   (Mouse, Keyboard)}
    \item \textbf{EXPO Events}: A list of approximately 35 events. \textit{RETROCOMPUTING $\rightarrow$ Event}
    \item \textbf{Software Relationships}: For each videogame, a \textit{producedBy} relationship has been established linking the software to the company that developed it.
    \item \textbf{Console Relationships}: For each console, the producing company has been recorded along with associated relationships to already existing consoles.
    \item \textbf{Peripheral Relationships}: For each mouse and keyboard device, the producer has been identified.
    \item \textbf{Geographical Data}: Inclusion of Matera and surrounding cities (approximately 30 locations).
    \item \textbf{Internet Protocols}: Updates include renaming 8 existing protocols and adding around 70 new entries.  \textit{RETROCOMPUTING $\rightarrow$ InternetProtocol}
    \item \textbf{Crapiata}: A traditional dish from Matera, described as a soup made with legumes and vegetables, albeit missing some ingredients.   \textit{FOOD}
    \item \textbf{Culinary Relationships}: Established relevant relationships associated with the aforementioned dish.
\end{itemize}

\subsubsection{Proposed Ontology Enhancements}

The following updates are suggested to further improve the ontology:

\begin{itemize}
    \item \textbf{Genre Classification for Videogames}:
    \begin{itemize}
        \item \textit{First Person Shooter}: Introduce as a subclass of \textit{Videogame}.
        \item \textit{Fighting Game}: Introduce as a subclass of \textit{Videogame}.
        \item \textit{Action Game}: Introduce as a subclass of \textit{Videogame}.
        \item \textit{Sandbox Game}: Introduce as a subclass of \textit{Videogame}.
    \end{itemize}
    \item \textbf{Software Development Relationships}: The \textit{developedBy} relation currently connects \textit{Software} to \textit{Person}, it is proposed to extend this relationship to include \textit{Organization} as well.
    \item \textbf{Operational Context}:
    \begin{itemize}
        \item Introduce a \textit{executedOn} relationship between \textit{Software} and \textbf{Operating System} to denote the platform on which the software operates.
        \item Introduce an \textit{runOn} relationship between \textit{Videogame} and \textbf{Console} to specify the console on which the videogame is executed.
    \end{itemize}
\end{itemize}

\subsubsection{Interface Improvements}

Several adjustments have been made to enhance the user interface:

\begin{itemize}
    \item Incorporation of an HTML Date Type field for the insertion of dates.
    \item Modification of the relationship creation process to allow starting from either the Subject or the Object.
\end{itemize}